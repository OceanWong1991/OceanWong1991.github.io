\documentclass[12pt]{article}
\usepackage[
  letterpaper,
  margin=1in,
  headsep=4pt, % separation between header rule and text
]{geometry}
\usepackage{xcolor}
\usepackage{fancyhdr}
\usepackage{tgschola}
\usepackage{lastpage}
\usepackage{setspace}
\renewcommand{\baselinestretch}{1.2}

\pagestyle{fancy}
\fancyhf{}
\fancyhead[C]{%
  \footnotesize\sffamily
  \yourname\quad
  web: \textcolor{black}{\itshape\yourweb}\quad
  \textcolor{black}{\youremail}}
% \fancyfoot[C]{Page \thepage\ of \pageref{LastPage}}

\newcommand{\soptitle}{Statement Of Intent}

\newcommand{\yourname}{Xuezhong Wang}
\newcommand{\youremail}{OceanWong1991@gmail.com}
\newcommand{\yourweb}{https://oceanwong1991.github.io/}

\newcommand{\statement}[1]{\par\medskip
  \underline{\textcolor{black}{\textbf{#1:}}}\space
}

\usepackage[
  colorlinks,
  breaklinks,
  pdftitle={\yourname - \soptitle},
  pdfauthor={\yourname},
  unicode
]{hyperref}

\setlength{\parindent}{0pt}
\begin{document}

\begin{center}\LARGE\soptitle\\
\large \yourname\ (Computer Science (MSc) applicant for Sep---2022)
\end{center}

\hrule
\vspace{1pt}
\hrule height 1pt

\sffamily
\mdseries

\bigskip
My name is Xuezhong Wang, I am passionate about Artificial Intelligence, data analysis especially video analytics. For years, most applications of computer version center on images, with less focus on sequences of images. However, the use of image based models fails to unlock the true potential of video analytics. And the limits of what is possible with frame-based methods is becoming ever so clear. The solution, involving the temporal dimension, unveils new information and opens the doors to a whole range of new possibilities, because sequences of images provide new information about action. For example, we can track an obstacle through a sequence of images and understand it’ behaviour to predict the next move. We can track a human pose, and understand the action taken with action classification.


\bigskip

I am looking forward to the possibility of working with professor Shan Du, who works within the broad areas of Artificial Intelligence, computer vision, and video surveillance system, I am confident that my learning and work experience matches her expectations very well, I got my B.S. degree in Information and Computing Science in June 2015 at HBAU. After graduation, I worked as an Android software engineer in a startup. After one year's hard work, I can develop programs independently without any difficulty, Thus, I decided to challenge myself by trying something more difficult. My commitment, learning ability and result-oriented strategy enable me to be a successful AI engineer after consistent self learning. I have more than four years of experience in Machine Learning, Deep Learning, and Computer Vision, and experience in handling a variety of projects from concept to completion. During my 3 years leadership in my team, I have thoroughly demonstrated my strong communication, problem solving and cooperation skills. Our team completed a variety of projects successfully and received much recognition from colleagues. Our ADAS projects, like LCA and BSD, have achieved great success in the market and have been exported overseas.

\bigskip

In my opinion, the video analytics is a multi-disciplinary area, which requires knowledge of object detection, movement tracking, behavioural pattern,  AI, GPUs, software engineering and a hundred other talents. I am excited about working with professor Shan Du and committing my passion and intelligence to this promising area which is full of infinite possibilities.

\bigskip

Thank you for your consideration!


\end{document}


