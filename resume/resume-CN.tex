% !TEX TS-program = xelatex
% !TEX encoding = UTF-8 Unicode
% !Mode:: "TeX:UTF-8"

\documentclass{resume}
\usepackage{zh_CN-Adobefonts_external} % Simplified Chinese Support using external fonts (./fonts/zh_CN-Adobe/)
% \usepackage{NotoSansSC_external}
% \usepackage{NotoSerifCJKsc_external}
% \usepackage{zh_CN-Adobefonts_internal} % Simplified Chinese Support using system fonts
\usepackage{linespacing_fix} % disable extra space before next section
\usepackage{cite}

\begin{document}
\pagenumbering{gobble} % suppress displaying page number

\name{王学忠}

\basicInfo{
  \email{OceanWong1991@gmail.com} \textperiodcentered\ 
  \phone{(+86) 186-1236-2194} \textperiodcentered\ 
  \homepage[OceanWong]{https://oceanwong1991.github.io/}}

\section{\faUsers\ 工作/项目经历}
\datedsubsection{\textbf{The University of British Columbia}  Okanagan}{2022年3月 -- 2022年12月}
\role{Research Assistant}

\datedsubsection{\textbf{北京新颖智通科技有限公司}  北京}{2020年6月 -- 2021年12月}
\role{高级算法工程师}

\datedsubsection{\textbf{北京双髻鲨科技有限公司}  北京}{2015年6月 -- 2021年2月}
\role{高级算法工程师}


\datedsubsection{\textbf{Leak gas detection}}{2022年3月 -- 2022年12月}
\role{Python, Darknet, Pytorch, Linux}{助研}
\begin{onehalfspacing}
人工智能在环境保护领域的应用,随着全球气候的变化,大家越来越越关注温室气体的泄漏问题,以前研究主要集中于使用传统的机器视觉方式建模分析(optical-flow based change detection algorithm,Fourier transform infrared),受限于特征提取的局限性,鲁棒性较差。对于环境变化,例如:风向,风速,室外温度;影响很大。因此随着人工智能的发展,我们寻求利用深度学习的方式解决对于泄漏气体的检测问题。


\begin{itemize}
  \item Background subtraction, 去除背景信息,让模型集中学习用用信息
  \item 
  \item 
\end{itemize}
\end{onehalfspacing}

\datedsubsection{\textbf{商用车盲区监测}}{2020年2月 -- 2021年2月}
\role{Python, Darknet, Pytorch, Linux}{项目负责人}
\begin{onehalfspacing}
商用车盲区检测系统(BSD)主要是对商用车盲区行人跟车辆的检测,当驾驶员变道或泊车时提供视觉或声音上的提示,帮助驾驶员安全驾驶.由于模型需要在低端车机芯片上运行,因此需要严格的控制模型大小并且保证检出结果的准确性.

\begin{itemize}
  \item 设计数据处理,分析模块.针对鱼眼视图数据设计出一套自动数据打标签系统,提升90\%+ 数据标注效率;根据回传数据主动分析模型盲点,针对性添加数据增强模型鲁棒性并最大化避免对有效数据的稀释
  \item 设计并训练主要的检测模型.(1.backbone 的变化, 2. YOLOV5 anchor匹配机制, 3.Mosic及Mixup 数据增广的使用, 4.YOLOX Decouple head:reg, cls, obj)
  \item 模型优化.为了使模型能够在低端芯片上运行(例如:全志T5),对模型进行了最大化剪枝,最终使模型的计算量由1.2BFlops减至0.139 BFlops,mAP由90\% 增加至 95\%
\end{itemize}
\end{onehalfspacing}

\datedsubsection{\textbf{ 巡航机器人}}{2019 年12月 -- 2020年 5月}
\role{Python, ROS}

\begin{onehalfspacing}
巡航机器人:使用雷达,IMU,Camera,Odometry等传感器,实现构建地图,主动避障,定点巡航功能
\begin{itemize}
  \item 设计机器人底盘
  \item 构建Unicycle Robot Model(一万向轮,两驱动轮). 并标定机器人底盘
  \item 部署,调试 google-cartographer, ros-navigation 算法.
\end{itemize}
\end{onehalfspacing}

\datedsubsection{\textbf{ 驾驶员疲劳检测系统(DSM)}}{2019 年2月 -- 2020年 12月}
\role{Python, caffe}

\begin{onehalfspacing}
DSM:使用红外相机图片对驾驶员行为进行检测分析(分神,打瞌睡,抽烟,打电话等)
\begin{itemize}
  \item 设计驾驶员人脸,关键点及行为检测神经网络
  \item 神经网络搭建,使用TensorFlow
  \item 训练网络,采用Online Hard Example Mining,数据增广等一系列技巧,将准确率从98\% 提升至 99.3\%.
  \item 部署并优化算法.采用一些传统算法及策略(例如:光流追踪)将检测帧率由5HZ 提升至 26HZ(Rock Chip RK3399)
\end{itemize}
\end{onehalfspacing}



\section{\faHeartO\ 获奖情况}
\begin{itemize}[parsep=0.5ex]
\item {\datedline{\textit{获得专家头衔(kaggle). TOP 0.67\%(1072/159,193)}}{2019.12}}
\item {\datedline{\textit{个人银牌, PB Top 6\%(30/546)} Lyft 3D Object Detection for Autonomous Vehicles Kaggle Competition}{ 2019.12}}
\item {\datedline{\textit{LB Top 7\%(182/2427); PB Top 13\%(301/2427)} Severstal: Steel Defect Detection Kaggle Competition}{2019.10}}
\item {\datedline{\textit{个人银牌, PB Top 2\%(55/2928).} APTOS 2019 Blindness Detection Competition Kaggle Competition}{ 2019.9}}
\item {\datedline{\textit{一等} 专业奖学金}{2014.4}}
\item {\datedline{\textit{二等奖 领队} 中国大学生数学建模竞赛(CUMCM)}{2013.9}}
\item {\datedline{\textit{一等奖 领队} 2013年东北三省数学建模联赛}{2013.5}}
\end{itemize}



\section{\faCogs\ IT 技能}
% increase linespacing [parsep=0.5ex]
\begin{itemize}[parsep=0.5ex]
  \item 编程语言: Python > Java > C++
  \item 平台: Linux,Docker
  \item 工具: Pytorch
\end{itemize}

\section{\faGraduationCap\  教育背景}
\datedsubsection{\textbf{黑龙江八一农垦大学}, 大庆, 黑龙江}{2011 -- 2015}
\textit{学士}\ 信息与计算科学



\section{\faInfo\ 其他}
% increase linespacing [parsep=0.5ex]
\begin{itemize}[parsep=0.5ex]
  \item Kaggle: https://www.kaggle.com/oceanwong
  \item GitHub: https://github.com/OceanWong1991
  \item homepage: https://oceanwong1991.github.io/
  \item 语言: 英语 - 熟练 IELTS 7.0
\end{itemize}

\end{document}
