% !TEX TS-program = xelatex
% !TEX encoding = UTF-8 Unicode
% !Mode:: "TeX:UTF-8"

\documentclass{resume}
\usepackage{zh_CN-Adobefonts_external} % Simplified Chinese Support using external fonts (./fonts/zh_CN-Adobe/)
% \usepackage{NotoSansSC_external}
% \usepackage{NotoSerifCJKsc_external}
% \usepackage{zh_CN-Adobefonts_internal} % Simplified Chinese Support using system fonts
\usepackage{linespacing_fix} % disable extra space before next section
\usepackage{cite}

\begin{document}
\pagenumbering{gobble} % suppress displaying page number

\name{王学忠}

\basicInfo{
  \email{OceanWong1991@gmail.com} \textperiodcentered\ 
  \phone{(+86) 186-1236-2194} \textperiodcentered\ 
  \homepage[OceanWong]{https://oceanwong1991.github.io/}}
 

% • 7+ years of experience in Machine Learning, Deep Learning, and Computer Vision
% • Research experience both in the industry and academic settings.
% • Experience in handling a variety of projects from concept to completion.
% • Strong leadership, communication, and teamwork skills.
% • Quick learning abilities and result oriented approach. 

\section{\faUsers\ 工作/项目经历}
\datedsubsection{\textbf{北京双髻鲨科技有限公司} 北京}{2015年5月 -- 至今}
\role{AI研发部经理}

\begin{itemize}
  \item 在机器学习,深度学习及计算机视觉有4年+ 的经验
  \item 主导过很多从概念到实现并落地的AI项目
  \item 较强的沟通,团队合作能力
  \item 很强的学习,调研能力;并以结果为导向解决工程性问题
\end{itemize}

\datedsubsection{\textbf{分布式科学上网姿势}}{2014年6月 -- 至今}
\role{Golang, Linux}{个人项目,和富帅糕合作开发}
\begin{onehalfspacing}
分布式负载均衡科学上网姿势, https://github.com/cyfdecyf/cow
\begin{itemize}
  \item 修复了连接未正常关闭导致文件描述符耗尽的 bug
  \item 使用Chord 哈希 URL, 实现稳定可靠地分流
  \item xxx (尽量使用量化的客观结果)
\end{itemize}
\end{onehalfspacing}

\datedsubsection{\textbf{\LaTeX\ 简历模板}}{2015 年5月 -- 至今}
\role{\LaTeX, Python}{个人项目}
\begin{onehalfspacing}
优雅的 \LaTeX\ 简历模板, https://github.com/billryan/resume
\begin{itemize}
  \item 容易定制和扩展
  \item 完善的 Unicode 字体支持,使用 \XeLaTeX\ 编译
  \item 支持 FontAwesome 4.5.0
\end{itemize}
\end{onehalfspacing}


\section{\faHeartO\ 获奖情况}
\datedline{\textit{第一名}, xxx 比赛}{2013 年6 月}
\datedline{其他奖项}{2015}

% Reference Test
%\datedsubsection{\textbf{Paper Title\cite{zaharia2012resilient}}}{May. 2015}
%An xxx optimized for xxx\cite{verma2015large}
%\begin{itemize}
%  \item main contribution
%\end{itemize}

\section{\faCogs\ IT 技能}
% increase linespacing [parsep=0.5ex]
\begin{itemize}[parsep=0.5ex]
  \item 编程语言: C == Python > C++ > Java
  \item 平台: Linux
  \item 开发: xxx
\end{itemize}

\section{\faGraduationCap\  教育背景}
\datedsubsection{\textbf{上海交通大学}, 上海}{2013 -- 至今}
\textit{在读硕士研究生}\ 信息与通信工程, 预计 2016 年 3 月毕业
\datedsubsection{\textbf{西安电子科技大学}, 西安, 陕西}{2009 -- 2013}
\textit{学士}\ 通信工程



\section{\faInfo\ 其他}
% increase linespacing [parsep=0.5ex]
\begin{itemize}[parsep=0.5ex]
  \item 技术博客: http://blog.yours.me
  \item GitHub: https://github.com/username
  \item 语言: 英语 - 熟练(TOEFL xxx)
\end{itemize}

%% Reference
%\newpage
%\bibliographystyle{IEEETran}
%\bibliography{mycite}
\end{document}
