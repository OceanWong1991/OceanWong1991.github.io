\documentclass{article}
\usepackage[
  letterpaper,
  margin=1in,
  headsep=4pt, % separation between header rule and text
]{geometry}
\usepackage{xcolor}
\usepackage{fancyhdr}
\usepackage{tgschola}
\usepackage{lastpage}
\usepackage{setspace}
\renewcommand{\baselinestretch}{1.2}

\pagestyle{fancy}
\fancyhf{}
\fancyhead[C]{%
  \footnotesize\sffamily
  \yourname\quad
  web: \textcolor{black}{\itshape\yourweb}\quad
  \textcolor{black}{\youremail}}
% \fancyfoot[C]{Page \thepage\ of \pageref{LastPage}}

\newcommand{\soptitle}{Statement Of Intent}

\newcommand{\yourname}{Wang Xuezhong}
\newcommand{\youremail}{OceanWong1991@gmail.com}
\newcommand{\yourweb}{https://oceanwong1991.github.io/}

\newcommand{\statement}[1]{\par\medskip
  \underline{\textcolor{black}{\textbf{#1:}}}\space
}

\usepackage[
  colorlinks,
  breaklinks,
  pdftitle={\yourname - \soptitle},
  pdfauthor={\yourname},
  unicode
]{hyperref}

\documentclass[12pt]{article}
\begin{document}

\begin{center}\LARGE\soptitle\\
\large \yourname\ (Computer Science (MSc) applicant for Sep---2022)
\end{center}

\hrule
\vspace{1pt}
\hrule height 1pt

% \bigskip
\rmfamily


My name is Wang Xuezhong, I am particularly passionate about Artificial Intelligence, VR especially metaverse. and I strongly believe that the metaverse is the next generation of the internet. As computers are getting better at gesture recognition, which will enable us to interact more naturally with computers, and eventually interpret and understand emotion and body language. AI plays an important role in this process, for example, Virtual reality needs to render the best information where your eye is focused. AI can be used to predict where your eye will look next, even through blinking, to help prepare the best rendering in advance. This is important for delivering the most immersive experience, at the same time, I think the ultimate interface should be personalized, and AI could make it possible. because everyone’s brain is different, so the role of AI is to learn from and adapt to the uniqueness of each individual. about the rendering, I believe The next generation of AI models will be capable of editing and generating images in response to text input, and hopefully, they’ll understand the text better because of the many images they’ve seen and more powerful chips they could use. finally, Artificial intelligence will power conversations, emotions, behaviors of characters you'll meet and even befriend in the metaverse.

I am looking forward to the possibility of working with professor Brandon Haworth, who works within the broad areas of Artificial Intelligence, Virtual Reality, Game Design and Development and Human-Computer Interaction. and he also encourages applicants from many different backgrounds and abilities. I think my experience matches his requirements very well, I got my B.S. degree in Information and Computing Science in June 2015 at HBAU. After graduation, I worked as an Android software engineer in a startup, After a year's hard work, I felt free to develop Android applications, So, I decided to challenge myself to do something much more difficult, so I started self-education in artificial intelligence, my quick learning abilities and result oriented approach enable me to be an AI engineer successfully. I have about  4+ years of experience in Machine Learning, Deep Learning, and Computer Vision, and experience in handling a variety of projects from concept to completion, working as a team leader in my team for about 3 years, I think I have strong leadership, communication, and teamwork skills. I can also use several programming languages ​​proficiently, including Java, Pathon.

In my opinion, the metaverse is a multi-disciplinary area, which requires knowledge of game design, game theory, behavioral pattern, analytics, databases, music, AI, GPUs, graphics, user experience design, storytelling, software engineering and a hundred other talents. professor Brandon Haworth's knowledge and my experience and curiosity can ensure me achieve something in this area.

Thank you for your consideration!


\end{document}
